\documentclass[11pt]{article}
\usepackage[T1]{fontenc}
\usepackage[utf8]{inputenc}
\usepackage{graphicx,xcolor,epstopdf,tgtermes,pdflscape,float,amsmath,amssymb,amsthm,textcomp,enumerate,multicol,tikz,geometry,fancyhdr,lastpage,hyperref}
\pdfminorversion=4

\input{InstructionsMacros.tex} % Loading the macros 

\geometry{total={210mm,297mm},
left=25mm,right=25mm,%
bindingoffset=0mm, top=20mm,bottom=20mm}
\linespread{1.3}

\newcommand{\linia}{\rule{\linewidth}{0.5pt}}
\makeatletter
\renewcommand{\maketitle}{
\begin{center}
\vspace{2ex}
{\huge \textsc{\@title}}
\vspace{1ex}
\\
\linia\\
\@author \hfill \@date
\vspace{4ex}
\end{center}
}
\makeatother

\pagestyle{fancy}
\lhead{}
\chead{}
\rhead{}
\lfoot{Laval Methods Workshop}
\cfoot{}
\rfoot{Page \thepage\ /\ \pageref*{LastPage}}
\renewcommand{\headrulewidth}{0pt}
\renewcommand{\footrulewidth}{0pt}

%%%----------%%%----------%%%----------%%%----------%%%
%%%----------%%%----------%%%----------%%%----------%%%

\begin{document}

\title{\R + \LaTeX: A Very Brief Introduction\\ Short guide for installation}
\author{Yannick Dufresne, Ph.D.}
\date{ }
\maketitle

\section{Necessary software installations for the workshop} % (fold)
%\label{sec: installation_des_logiciels_n_cessaires_pour_le_tp_et_le_reste_du_cours_}

    \subsection*{Install \LaTeX} % (fold)
        \begin{itemize}
          \item Go to the web page \url{http://latex-project.org/ftp.html} and download the version of \LaTeX relevant for your operating system. Install the software downloaded.
        \end{itemize}

    \subsection*{Install \R} % (fold)
        \begin{itemize}
          \item Go to the web page \url{http://cran.rstudio.com/} and download the version of \R relevant for your operating system. Install the software downloaded.
        \end{itemize}

    \subsection*{Install RStudio} % (fold)
        \begin{itemize}
          \item Go to the web page \url{http://www.rstudio.com/products/rstudio/download/} and download the version of RStudio relevant for your operating system in section ``\textbf{Installers for Supported Platforms}'' at the bottom of the page. Install the software downloaded.
        \end{itemize}
        
\newpage

\section{Open the necessary files in RStudio} % (fold)
\label{sec:rstudio}

    \subsection{The file containing the .R code} % (fold)
    %\label{sub:le_document_}
        % subsection le_document_ (end)
        \begin{enumerate}
          \item In RStudio, open the file named \textbf{``''} that you have previously received by email.          
          \item Do not modify anything in the file for now.
        \end{enumerate}

    \subsection{The file containing the .tex code} % (fold)
    %\label{sub:le_document_latex_contenant_le_fichier_}
        \begin{enumerate}
            \item In RStudio, open the file named \textbf{``''} that you have previously received by email.          
            \item Do not modify anything in the file for now.

            %\begin{tips}
                To open a file in RStudio, click on ``File' and then ``Open File'' and choose the location where you have unzip the file \R on your hardware.
                \begin{figure}[H]
                	\centering
                	%\includegraphics[width=0.3\textwidth]{open}
                \end{figure}
                You could also use the the shortcut ``CTRL+o'' on Windows or ``CMD+o'' on Mac OS.
            %\end{tips}
        \end{enumerate}
    % section rstudio (end)

\newpage

    In the window located at the top left of the RStudio interface, you should see two new tabs that emerged from the opening of the two files ``\textbf{}'' and ``\textbf{}''.

    \begin{enumerate}
        \item Click on ``\textbf{}'' to display the content of the \R file.
        \end{enumerate}

        %\begin{tips}
            All the lines that are begining with ``\#'' are considered as comments in a code written in \R language. It is important to pay attention to those specific characters to be able to better understand the code and modify it more easily.
        %\end{tips}
        
\section{\R Code: Let the fun begin...} % (fold)
%\label{sec:jouer_avec_le_code_r}

\end{document}
